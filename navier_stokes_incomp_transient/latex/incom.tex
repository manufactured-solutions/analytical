\documentclass{article}
\title{\bf{Incompressible Turbulence}}
\author{Nicholas Malaya} \date{}

\begin{document}
\maketitle

%%%%%%%%%%%%%%%%%%%%%%%%%%%%%%%%%%%%%%%%%%%%%%%%%
%
%        model
%
%%%%%%%%%%%%%%%%%%%%%%%%%%%%%%%%%%%%%%%%%%%%%%%%%
\newpage
\section{Formulation}


Starting with the Navier-Stokes equations:
\begin{equation}
  \frac{\partial u_i}{\partial t} = -\frac{\partial P}{\partial x_i} + H_i + \frac{1}{Re}\nabla^2 u_i
\end{equation}
With H being the convective terms.

In addition, we are dealing with incompressible turbulence,
\begin{equation}
  \frac{\partial u_i}{\partial x_i} = 0
\end{equation}
We Define: 
\begin{equation}
  \nabla^2 v(y) = \phi(y)
\end{equation}
and, 
\begin{equation}
  \nabla \times v(y) = g(y)
\end{equation}
We can now reduce the NS equations to:

\begin{equation}
  \frac{\partial }{\partial t} \nabla^2 v = h_v + \frac{1}{Re}\nabla^4 v
\end{equation}
and,
\begin{equation}
  \frac{\partial}{\partial t} g = h_g + \frac{1}{Re}\nabla^2 g
\end{equation}
Where we have defined hv as, 
\begin{equation}
  h_v=-\frac{\partial}{\partial y}(\frac{\partial H_1}{\partial x} + \frac{\partial H_3}{\partial z}) + (\frac{\partial^2}{\partial x^2} + \frac{\partial^2}{\partial z^2})H_2
\end{equation}
and hg is,
\begin{equation}
  h_g= \frac{\partial H_1}{\partial z} - \frac{\partial H_3}{\partial x}
\end{equation}
Notice that the derivatives with respect to Z and X are trivial to compute, since those velocities are in wavespace. However, hv requires a derivative with respect to y, which is more complex. 
(Typically Chebyshev polynomials, compact finite difference, or B-splines are used to evaluate this derivative)

\newpage
\section{Numerical Procedure}

This section is based around a third-order low storage Runge-Kutta (RK3) method (see ''Spalart Moser Rogers, J. Computational Physics, (1991) vol. 96 pp. 297-324'' )

At this time we discretize our equations for time advancement. RK3 treats the nonlinear terms explicitly, and the viscous terms implicitly.

The RK3 substeps go as,
\begin{equation}
  \begin{array}{l l}
    u'=u^0   + \Delta t[L(\alpha_1 u^0 + \beta_1 u') + \gamma_1 N^0] \\
    u''=u'   + \Delta t[L(\alpha_2 u' + \beta_2 u'') + \gamma_2 N^1 + \xi_1 N^0] \\
    u'''=u'' + \Delta t[L(\alpha_3 u'' + \beta_3 u''') + \gamma_3 N^2 + \xi_2 N^1] \\ \end{array}
\end{equation}
Or, in general, 
\begin{equation}
  \frac{u^{n+1} - u^{n}}{\Delta t} = L(\alpha_{n+1} u^n + \beta_{n+1} u^{n+1}) + \gamma_{n+1} N^n + \xi_{n} N^{n-1}
\end{equation}
Where the variables alpha, beta, gamma and xi are,

%% \begin{equation} 
%%   \begin{tabular}{| c | c | c | c |}
%%     \alpha_1 =  \frac{29}{96} & \beta_1 =  \frac{37}{160} & \gamma_1 =  \frac{8}{15} & \xi_1 = 0 \\
%%     \alpha_2 = -\frac{3}{40}  & \beta_2 =  \frac{5}{24}   & \gamma_2 =  \frac{5}{12} & \xi_2 = -\frac{17}{60} \\
%%     \alpha_3 =  \frac{1}{6}   & \beta_3 =  \frac{1}{6}    & \gamma_3 =  \frac{3}{4}  & \xi_3 = -\frac{5}{12}  \\
%%   \end{tabular}
%% \end{equation}

Notice that,
\begin{equation} 
  \partial_t g = h_g + \frac{1}{Re}\nabla^2 g
\end{equation}
is, (for the first substep)
\begin{equation} 
  \frac{u^1-u^0}{\Delta t} =  L(\alpha_1 u^0 + \beta_1 u^1) + \gamma_1 N^0
\end{equation}
We can rearrange this as,
\begin{equation} 
  u^1= u^0 + \Delta t[L( \alpha_1 u^0 + \frac{\beta_1}{Re} \nabla^2 u^1) + \gamma_1 N^0]
\end{equation}
Or,
\begin{equation} 
  (1 - \frac{\beta_1 \Delta t}{Re} \nabla^2) u^1= u^0 + \Delta t[ \alpha_1 L(u^0) + \gamma_1 N^0]
\end{equation}
Multiply both sides by the negative Reynolds number divided by Beta times the timestep,

\begin{eqnarray} 
    ((-1)(\frac{Re}{\beta_1 \Delta t}  - \nabla^2) u^1= - \frac{Re}{\beta_1 \Delta t} (u^0 + \Delta t[ \alpha_1 L(u^0) + \gamma_1 N^0]) \\
    \nabla^2 = \partial_{yy} - k_x^2 - k_y^2 \\
    Rk_i = \frac{Re}{\beta_i \Delta t}
\end{eqnarray}

Then,
\begin{equation}   
  (\partial_{yy} - Rk_1 - k_x^2 - k_y^2) u^1= - Rk_1 (u^0 + \Delta t [ \alpha_1 \nabla^2 u^0 + \gamma_1 h_g^0]) \\
\end{equation}

Generalized for the nth substep, this becomes,
\begin{equation} 
  (\partial_{yy} - Rk_{n+1} - k_x^2 - k_y^2) u^{n+1} = - Rk_{n+1} (u^n + \Delta t[ \alpha_{n+1} (\partial_{yy} - k_x^2 - k_z^2) u^{n} + \gamma_{n+1} h_g^{n} + \xi_{n} h_g^{n-1}]) \\
\end{equation}

\newpage
\section{Unit tests}

We test our implementation of this method by the ''method of manufactured solution''. This is useful for any problem that does 
not have obvious (or possibly, does not possess) analytical solutions. To use this method, we choose a solution that is 
capable of satisfying the equations, find the required inputs, and then use this as inputs for the numerical code. 

We choose a suitable polynomial as our 'manufactured solution' that satisfies the boundary conditions:
\begin{equation} 
  v(y) = P(y)(1-y)^2(1+y)^2
\end{equation}

\begin{equation} 
  P(y) = y^5 + y^4 + y^3 + y^2 + y \\
\end{equation}

%Where the derivatives are equal to, 
%\begin{equation} 
%  \begin{array}
%    av'(y) = (1 - y)^2 (1 + y)^2 (1+2y+3y^2+4y^3+5y^4) \\
%    + 2(1-y)^2(1+y)(y+y^2+y^3+y^4+y^5) \\
%    -2(1-y)(1+y)^2(y+y^2+y^3+y^4+y^5) \\
%  \end{array}
%\end{equation}

\end{document}
