\documentclass[10pt]{article}
\usepackage[utf8x]{inputenc}
\usepackage{amsmath}
\usepackage{geometry}
%\usepackage[mathcal]{euscript}
\geometry{ top=2.5cm, bottom=2cm, left=2cm, right=2cm}
\usepackage[authoryear]{natbib}
\usepackage{pdflscape}
\usepackage{graphicx}
\usepackage{subcaption}
%\geometry{papersize={216mm,330mm}, top=3cm, bottom=2.5cm, left=4cm,  right=2cm}

\newcommand{\D}{\partial}
\newcommand{\Diff}[2] {\dfrac{\partial( #1)}{\partial #2}}
\newcommand{\diff}[2] {\dfrac{\partial #1}{\partial #2}}
\newcommand{\bv}[1]{\ensuremath{\mbox{\boldmath$ #1 $}}}
\newcommand{\gv}[1]{\ensuremath{\mbox{\boldmath$ #1 $}}}% for vectors of Greek letters
\newcommand{\grad}[1]{\gv{\nabla} #1}
\newcommand{\Rho}{\,\mathtt{Rho}}
\newcommand{\PP}{\,\mathtt{P}}
\newcommand{\U}{\,\mathtt{U}}
\newcommand{\V}{\,\mathtt{V}}
\newcommand{\W}{\,\mathtt{W}}
\newcommand{\Lo}{\,\mathcal{L}}
\newcommand{\pfrac}[2]{\frac{\partial#1}{\partial#2}}
\newcommand{\wt}[1]{\widetilde{#1}}
\newcommand{\ud}{\,\mathrm{d}}
%opening
\title{Manufactured Solution for the Compressible Navier--Stokes Equations with Large Eddy Simulation models using automatic differentiation}
\author{Marc T. Henry de Frahan}

\begin{document}

\maketitle

\begin{abstract}
  In this document, we describe the equations used for the generation
  of the source terms for the Navier-Stokes with Large Eddy Simulation
  models. We will use automatic differentiation for the source term
  derivation.
\end{abstract}

\section{Mathematical Model}
The conservation of mass, momentum, and total energy for the Favre-filtered compressible viscous fluid may be written as:
\begin{equation}
  \label{eq:ns_01}
  \pfrac{\bar{\rho}}{t} + \pfrac{}{x_j}\left( \bar{\rho} \wt{u}_j \right) = 0,
\end{equation}

\begin{equation}
  \label{eq:ns_02}
  \pfrac{\bar{\rho} \wt{u}_i}{t} + \pfrac{}{x_j}\left( \bar{\rho} \wt{u}_i \wt{u}_j + \bar{p} \delta_{ij} - \wt{\sigma}_{ji} - \tau_{ji} \right) = 0,
\end{equation}

\begin{equation}
  \label{eq:ns_03}
  \pfrac{\bar{\rho} \wt{E}}{t} + \pfrac{}{x_j}\left( \left(\bar{\rho} \wt{E} + \bar{p}\right) \wt{u}_j + \wt{q}_j + \gamma c_v \mathcal{Q}_j - \wt{\sigma}_{ij} \wt{u}_i - \mathcal{J} \right) = 0.
\end{equation}

The resolved variables are denoted by an overbar
\begin{equation}
  \bar{f} = \int_D f(\bv{x'}) G(\bv{x}, \bv{x'}; \bar{\Delta}) \ud \bv{x'}
\end{equation}
where $D$ is the domain, $G$ is the filter, $\bar{\Delta}$ is the
filter width. For compressible flows, we use Favre-filtering, where
the variable is $\wt{f} = \frac{\overline{\rho f}}{\bar{\rho}}$.
Equations (\ref{eq:ns_01})--(\ref{eq:ns_03}) are known as
Favre-filtered Navier--Stokes equations and, $\rho$ is the density,
$\bv{u}=(u,v,w)$ is the velocity in $x$, $y$ or $z$-direction,
respectively, and $p$ is the pressure. For a calorically perfect gas,
these equations are closed with two auxiliary relations for energy:
\begin{equation}
  \label{eq:ns_04}
  E = e + \dfrac{u_i u_i}{2},\quad\mbox{and}\quad e=\dfrac{1}{\gamma -1}RT ,
\end{equation}
and with the ideal gas equation of state:
\begin{equation}
  \label{eq:ns_05}
  p=\rho RT.
\end{equation}

The diffusive fluxes
\begin{align}
  \wt{\sigma}_{ij} &= 2 \wt{\mu} \wt{S}_{ij} - \frac{2}{3} \wt{\mu} \delta_{ij}  \wt{S}_{kk},\\
  \wt{q}_j &= - \wt{k} \pfrac{\wt{T}}{x_j},
\end{align}
where $\wt{S}_{ij} = \frac{1}{2} \left( \pfrac{\wt{u}_i}{x_j} +
\pfrac{\wt{u}_j}{x_i}\right)$ is the strain rate tensor, $\wt{\mu}$
and $\wt{k}$ are the viscosity and termal conductivity for the
filtered temperature $\wt{T}$.

The sub-filter terms for the SFS stresses, SFS heat flux, and SFS
turbulent diffusion, are
\begin{align}
  \tau_{ij} &= \bar{\rho} \left(\wt{u_i u_j} - \wt{u_i}\wt{u_j}\right),\\
  \mathcal{Q}_j &= \bar{\rho} \left(\wt{u_j T} - \wt{u_j}\wt{T}\right),\\
  \mathcal{J}_j &= \bar{\rho} \left(\wt{u_j u_k u_k} - \wt{u_j}\wt{u_k u_k}\right),
\end{align}
and need to be modeled. In this work, we choose to model the SFS terms
using the standard Smagorinsky model.

The SFS stresses are modeled as
\begin{align}
  \tau_{ij} - \frac{\delta_{ij}}{3} \tau_{kk} &= 2 \mu_t \left(\wt{S}_{ij} - \frac{\delta_{ij}}{3}  \wt{S}_{kk} \right)
\end{align}
where $\mu_t = C_s^2 \bar{\Delta}^2 \bar{\rho} |\wt{S}|$, $\tau_kk = 2
C_I \bar{\Delta}^2 \bar{\rho} |\wt{S}|^2$, and $|\wt{S}| = \sqrt{2
  \wt{S}_{ij}\wt{S}_{ij}}$.

The SFS heat flux is modeled as
\begin{align}
  \mathcal{Q}_j = - \frac{\mu_t}{Pr_t} \pfrac{\wt{T}}{x_j}
\end{align}

The SFS turbulent diffusion is modeled as
\begin{align}
  \mathcal{J}_j = \wt{u}_k \tau_{jk}
\end{align}

The source terms for these equations are obtained through automatic
differentiation as implemented in MASA and METAPHYSICL.

\section{Results}

The method of manufactured solutions was used to verify the PeleC code
at the National Renewable Energy Laboratory. PeleC is a second order
finite volume code used in combustion applications. For these cases,
the Reynolds, Mach, and Prandtl numbers were set to 1 to ensure that
the different physics were equally important (viscosity, conductivity,
and bulk viscosity are non-zero and determined by the appropriate
non-dimensional number). The CFL condition was fixed to 0.1 to ensure
that the predictor-corrector time stepping method found a solution to
the system of equations. The initial solution was initialized to the
exact solution. Periodic boundaries are imposed everywhere. The Large
Eddy Simulation (LES) constants, $C_s$ and $C_I$, were chosen such
that the turbulent eddy viscosity was comparable to the viscosity,
i.e. $\frac{\mu_t}{\mu} = \mathcal{O}(1)$. Since the model scales with
the mesh spacing, $C_s$ and $C_I$ were scaled inversely with the mesh
spacing for the mesh refinement studies. For example, $C_s$ is set to
2 for the $8^3$ mesh and set to 4 for the $16^3$ mesh (for $C_I$, it
is 1 and 4, respectively). A convergence study shows second order for
Pele's treatment of the compressible Navier-Stokes equations with the
constant Smagorinsky Large Eddy Simulation model,
Figure\,\ref{fig:results}.

\begin{figure}%
  \centering%
  \begin{subfigure}[b]{0.49\textwidth}%
    \includegraphics[width=\textwidth]{./rho_error.png}
    \caption{Density.}%
    \label{fig:}%
  \end{subfigure}%
  \begin{subfigure}[b]{0.49\textwidth}%
    \includegraphics[width=\textwidth]{./u_error.png}%
    \caption{$u_1$.}%
    \label{fig:}%
  \end{subfigure}\hfill%
  \begin{subfigure}[b]{0.49\textwidth}%
    \includegraphics[width=\textwidth]{./v_error.png}%
    \caption{$u_2$.}%
    \label{fig:}%
  \end{subfigure}\hfill%
  \begin{subfigure}[b]{0.49\textwidth}%
    \includegraphics[width=\textwidth]{./w_error.png}%
    \caption{$u_3$.}%
    \label{fig:}%
  \end{subfigure}\hfill%
  \begin{subfigure}[b]{0.49\textwidth}%
    \includegraphics[width=\textwidth]{./p_error.png}%
    \caption{Pressure.}%
    \label{fig:}%
  \end{subfigure}\hfill%
  \caption{$L_2$ error as a function of $N$, the number of elements per side of the cube.}%
  \label{fig:results}%
\end{figure}%

\end{document}

