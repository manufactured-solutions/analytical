\documentclass[10pt]{article}
\usepackage[utf8x]{inputenc}
\usepackage{amsmath}
\usepackage{geometry}
\geometry{ top=3cm, bottom=2.5cm, left=2.5cm, right=2.5cm}
\usepackage[authoryear]{natbib}
\usepackage{pdflscape}
%\geometry{papersize={216mm,330mm}, top=3cm, bottom=2.5cm, left=4cm,  right=2cm}

\newcommand{\D}{\partial}
\newcommand{\Diff}[2] {\dfrac{\partial( #1)}{\partial #2}}
\newcommand{\diff}[2] {\dfrac{\partial #1}{\partial #2}}
\newcommand{\Lo}{\,\mathcal{L}}
\newcommand{\U}{\,\mathtt{U}}
\newcommand{\V}{\,\mathtt{V}}
%opening
\title{Manufactured Solution for 2D Burgers equations using Maple\footnote{Work based on \citet*{Salari_Knupp_2000}.}}
\author{Kémelli C. Estacio-Hiroms}

\begin{document}

\maketitle

\begin{abstract}
This document describes the usage of the Method of Manufactured Solutions (MMS) for Code Verification of Burgers solvers for two-dimensional, viscous and inviscid flows in both steady and unsteady regimen. The resulting source terms for each flow regimen are also presented.
\end{abstract}


\section{2D Burgers Equations}

Burgers' equation is a useful test case for numerical methods due to its simplicity and predictable dynamics, together with its non-linearity and multidimensionality. The various kinds of Burgers equation constitute a good benchmark to modeling traffic flows, shock waves and acoustic transmission, and they are also considered a basic model of nonlinear convective-diffusive phenomena such as those that arise
in Navier–Stokes equations.


The 2D Burgers  equations are:
\begin{equation}
 \label{eq:burgers2d_01}
\begin{split}
&\diff{ u}{t} + \diff{ u^2 }{x}+\diff{uv}{y}=\nu \left( \diff{^2u}{ x^2}+ \diff{^2u }{y^2}\right),\\
& \diff{ v}{t}+ \diff{ u v}{x} + \diff{  v^2 }{y}=\nu \left( \diff{^2v}{ x^2}+ \diff{^2v }{y^2}\right),
\end{split}
\end{equation}
%
where $u$ and $v$ are the velocity in the  $x$ and $y$  directions, respectively, and $\nu$ is the viscosity.

\section{Manufactured Solution}
The Method of Manufactured Solutions (MMS) provides a general procedure for  code accuracy verification \citep{Roache2002,Bond2007}.
The MMS constructs a non-trivial but analytical solution for the flow variables; this manufactured
solution usually does not  satisfy the governing equations, since the choice is somewhat arbitrary. However, by passing the solution through the governing equations gives the production terms $Q$. A modified set of equations formed by adding these source terms to the right-hand-side of the original governing equations is forced to become a model for the constructed solution, i.e., the manufactured solutions chosen \textit{a priori} are the analytical solutions of the MMS-modified governing equations.

Although the form of the manufactured solution is slightly arbitrary, it should be chosen to be smooth, infinitely differentiable and realizable. Solutions should be avoided which have negative densities, pressures, temperatures, etc. \citep{Salari_Knupp_2000,Roy2004}. Solutions should also be chosen that are sufficiently general so as to exercise all terms in the governing equations. Examples of manufactured solutions and convergence studies for Burgers, Euler and/or Navier--Stokes equations may be found in \citet{Salari_Knupp_2000,Roy2002,Roy2004,Bond2007,Orozco2010}.

\citet{Roy2002} introduce the general form of the primitive manufactured solution variables to be  a function of sines and cosines for the spatial variables only. In this work, \citet{Roy2002}'s manufactured solutions are modified in order to address temporal accuracy as well:
\begin{equation*}
 \label{eq:manufactured01}
  \phi (x,y,t) = \phi_0+ \phi_x\, f_s \left(\frac{a_{\phi x} \pi x}{L} \right) +  \phi_y \,f_s\left(\frac{a_{\phi y} \pi y}{L}\right) +  \phi_t \,f_s\left(\frac{a_{\phi t} \pi t}{L}\right),
\end{equation*}
where $\phi=u$ or $v$, and $f_s(\cdot)$ functions denote either sine or cosine function. Note that in this case, $\phi_x$, $\phi_y$,  and $\phi_t$ are constants  and the subscripts do not denote differentiation.

Therefore, a suitable set of time-dependent manufactured analytical solutions for  each one of the variables in Burgers equations is:
\begin{equation}
\begin{split}
\label{eq:manufactured02}
u\left(x,y,t\right) &= u_{0}+u_{x} \sin\left(\frac{a_{u x} \pi x}{L}\right)+u_{y} \cos\left(\frac{a_{u y} \pi y}{L}\right) + u_t \cos\left(\dfrac{a_{u t} \pi t}{L}\right),\\
v\left(x,y,t\right) &= v_{0}+v_{x} \cos\left(\frac{a_{v x} \pi x}{L}\right)+v_{y} \sin\left(\frac{a_{v y} \pi y}{L}\right)+ v_t \sin\left(\dfrac{a_{v t} \pi t}{L}\right),\\
\end{split}
\end{equation}



The source terms for the transient/steady and viscous/inviscid variations of the 2D Burger equations (\ref{eq:burgers2d_01}) using manufactured solutions  for $u$ and $v$, described in Equations (\ref{eq:manufactured02}) are presented in the following sections. Note that Equations (\ref{eq:manufactured02}) may be modified to address the steady case by setting $u_t=v_t=0$:
\begin{equation}
\begin{split}
\label{eq:manufactured03}
u\left(x,y\right) &= u_{0}+u_{x} \sin\left(\frac{a_{u x} \pi x}{L}\right)+u_{y} \cos\left(\frac{a_{u y} \pi y}{L}\right) ,\\
v\left(x,y\right) &= v_{0}+v_{x} \cos\left(\frac{a_{v x} \pi x}{L}\right)+v_{y} \sin\left(\frac{a_{v y} \pi y}{L}\right).
\end{split}
\end{equation}


\section{Transient Viscous Burgers Equation}

The transient, viscous version of Burgers equations is:
\begin{equation}
 \label{eq:burgers2d_tv}
\begin{split}
 &\diff{ u}{t} + \diff{ u^2 }{x}+\diff{uv}{y}=\nu \left( \diff{^2u}{ x^2}+ \diff{^2u }{y^2}\right),\\
&\diff{ v}{t}+ \diff{ u v}{x} + \diff{  v^2 }{y}=\nu \left( \diff{^2v}{ x^2}+ \diff{^2v }{y^2}\right),
\end{split}
\end{equation}
%
where $u$ and $v$ are the velocity in the  $x$ and $y$  directions, respectively, and $\nu$ is the viscosity.


For the generation of the analytical source term $Q_u$ for the velocity in the $x$-direction, the first component of Equation  (\ref{eq:burgers2d_tv}) is written as an  operator $\Lo$:
\begin{equation*}
 \label{eq:burgers2d_12}
\Lo=\diff{ u}{t} + \diff{ u^2 }{x}+\diff{uv}{y}-\nu \left( \diff{^2u}{ x^2}+ \diff{^2u }{y^2}\right),
\end{equation*}
which, when operated in Equation (\ref{eq:manufactured02}), provides source term $Q_{u}$:
\begin{equation}
\begin{split}\label{sourceQu_complete}
Q_u = &- \dfrac{a_{ut} \pi u_t}{L}\sin\left(\dfrac{a_{ut} \pi t}{L}\right)+\\
 &-\dfrac{a_{uy} \pi u_y \V_\text{trans} }{L}\sin\left(\dfrac{a_{uy} \pi y}{L}\right)+\\
 &+\dfrac{\pi \U_\text{trans}}{L}\left[2 a_{ux} u_x \cos\left(\dfrac{a_{ux} \pi x}{L}\right)+a_{vy} v_y \cos\left(\dfrac{a_{vy} \pi y}{L}\right)\right] +\\ &+ \dfrac{a_{ux}^2 \pi^2 u_x \nu}{L^2}\sin\left(\dfrac{a_{ux} \pi x}{L}\right)+ \dfrac{a_{uy}^2 \pi^2 u_y \nu}{L^2}\cos\left(\dfrac{a_{uy} \pi y}{L}\right).
\end{split}
 \end{equation}
where
\begin{equation}
 \begin{split} \label{aux_var_transient}
  \U_\text{trans} &= u_{0}+u_{x} \sin\left(\frac{a_{u x} \pi x}{L}\right)+u_{y} \cos\left(\frac{a_{u y} \pi y}{L}\right) + u_t \cos\left(\dfrac{a_{u t} \pi t}{L}\right),\\
\V_\text{trans} &= v_{0}+v_{x} \cos\left(\frac{a_{v x} \pi x}{L}\right)+v_{y} \sin\left(\frac{a_{v y} \pi y}{L}\right)+ v_t \sin\left(\dfrac{a_{v t} \pi t}{L}\right),\\
 \end{split}
\end{equation}

Analogously, for the generation of the analytical source term $Q_v$ for the $y$-velocity, the second component of  Equation~(\ref{eq:burgers2d_tv}) is written as an  operator $\Lo$:
\begin{equation}
  \label{eq:burgers2d_13}
  \Lo = \diff{ v}{t}+ \diff{ u v}{x} + \diff{  v^2 }{y}-\nu \left( \diff{^2v}{ x^2}+ \diff{^2v }{y^2}\right).
\end{equation}
and then applied to Equation  (\ref{eq:manufactured02}). It yields:
\begin{equation}
\begin{split}
Q_v &= \dfrac{a_{vt} \pi v_t }{L}\cos\left(\dfrac{a_{vt} \pi t}{L}\right)+\\
 &-\dfrac{a_{vx} \pi v_x \U_\text{trans} }{L}\sin\left(\dfrac{a_{vx} \pi x}{L}\right)+\\
 &+\dfrac{\pi \V_\text{trans}}{L}\left[a_{ux} u_x \cos\left(\dfrac{a_{ux} \pi x}{L}\right)+2 a_{vy} v_y \cos\left(\dfrac{a_{vy} \pi y}{L}\right)\right] +\\  &+\dfrac{a_{vx}^2 \pi^2 v_x \nu }{L^2}\cos\left(\dfrac{a_{vx} \pi x}{L}\right)+\dfrac{a_{vy}^2 \pi^2 v_y \nu }{L^2}\sin\left(\dfrac{a_{vy} \pi y}{L}\right).
\end{split}
\end{equation}
where  $\U_\text{trans}$ and $\V_\text{trans}$ are defined in  Equation (\ref{aux_var_transient}).


\section{Steady Viscous Burgers Equation}
%The steady solution to Burgers equation (\ref{eq:burgers2d_01}) and (\ref{eq:burgers2d_02}) is obtained by setting the $\omega$ to zero in the manufactured solution (\ref{eq:manufactured02}). Accordingly, the resulting source terms are:
For the steady viscous Burgers flow:
\begin{equation}
 \label{eq:burgers2d_sv}
\begin{split}
& \diff{ u^2 }{x}+\diff{uv}{y}=\nu \left( \diff{^2u}{ x^2}+ \diff{^2u }{y^2}\right),\\
 &\diff{ u v}{x} + \diff{  v^2 }{y}=\nu \left( \diff{^2v}{ x^2}+ \diff{^2v }{y^2}\right),
\end{split}
\end{equation}
the  time-independent  manufactured solution (\ref{eq:manufactured03}) is used. Then, Equation~(\ref{eq:burgers2d_sv}) is written as an  operator and passed through the manufactured solution. The resulting source terms are:
\begin{equation}
\begin{split}
Q_u = &-\dfrac{a_{uy} \pi u_y \V_\text{steady} }{L}\sin\left(\dfrac{a_{uy} \pi y}{L}\right)+\\
 &+\dfrac{\pi \U_\text{steady}}{L}\left[2 a_{ux} u_x \cos\left(\dfrac{a_{ux} \pi x}{L}\right)+a_{vy} v_y \cos\left(\dfrac{a_{vy} \pi y}{L}\right)\right] +\\
 &+ \dfrac{a_{ux}^2 \pi^2 u_x \nu}{L^2}\sin\left(\dfrac{a_{ux} \pi x}{L}\right)+ \dfrac{a_{uy}^2 \pi^2 u_y \nu}{L^2}\cos\left(\dfrac{a_{uy} \pi y}{L}\right).
\end{split}
 \end{equation}
and
\begin{equation}
\begin{split}
Q_v = &-\dfrac{a_{vx} \pi v_x \U_\text{steady} }{L}\sin\left(\dfrac{a_{vx} \pi x}{L}\right)+\\
 &+\dfrac{\pi \V_\text{steady}}{L}\left[a_{ux} u_x \cos\left(\dfrac{a_{ux} \pi x}{L}\right)+2 a_{vy} v_y \cos\left(\dfrac{a_{vy} \pi y}{L}\right)\right] +\\
 &+\dfrac{a_{vx}^2 \pi^2 v_x \nu }{L^2}\cos\left(\dfrac{a_{vx} \pi x}{L}\right)+\dfrac{a_{vy}^2 \pi^2 v_y \nu }{L^2}\sin\left(\dfrac{a_{vy} \pi y}{L}\right).
\end{split}
\end{equation}
where
\begin{equation}
 \begin{split} \label{aux_var_steady}
  \U_\text{steady} &= u_{0}+u_{x} \sin\left(\frac{a_{u x} \pi x}{L}\right)+u_{y} \cos\left(\frac{a_{u y} \pi y}{L}\right),\\
\V_\text{steady} &= v_{0}+v_{x} \cos\left(\frac{a_{v x} \pi x}{L}\right)+v_{y} \sin\left(\frac{a_{v y} \pi y}{L}\right).
 \end{split}
\end{equation}



\section{Transient Inviscid Burgers Equation}

The 2D transient inviscid Burgers  equations are:
\begin{equation}
 \label{eq:burgers2d_twv}
\begin{split}
 &\diff{ u}{t} + \diff{ u^2 }{x}+\diff{uv}{y}=0,\\
& \diff{ v}{t}+ \diff{ u v}{x} + \diff{  v^2 }{y}=0.
\end{split}
\end{equation}
%
Analogously to the transient viscous case, Equations (\ref{eq:burgers2d_twv}) are written as  operators $\Lo$ and applied to the manufactured analytical solutions (\ref{eq:manufactured02}). The source terms $Q_u$ and $Q_v$ are:
\begin{equation}
\begin{split}\label{sourceQu_twv}
Q_u = &- \dfrac{a_{ut} \pi u_t}{L}\sin\left(\dfrac{a_{ut} \pi t}{L}\right)+\\
 &-\dfrac{a_{uy} \pi u_y \V_\text{trans} }{L}\sin\left(\dfrac{a_{uy} \pi y}{L}\right)+\\
 &+\dfrac{\pi \U_\text{trans}}{L}\left[2 a_{ux} u_x \cos\left(\dfrac{a_{ux} \pi x}{L}\right)+a_{vy} v_y \cos\left(\dfrac{a_{vy} \pi y}{L}\right)\right].
\end{split}
 \end{equation}
and
\begin{equation}
\begin{split}
Q_v &= \dfrac{a_{vt} \pi v_t }{L}\cos\left(\dfrac{a_{vt} \pi t}{L}\right)+\\
 &-\dfrac{a_{vx} \pi v_x \U_\text{trans} }{L}\sin\left(\dfrac{a_{vx} \pi x}{L}\right)+\\
 &+\dfrac{\pi \V_\text{trans}}{L}\left[a_{ux} u_x \cos\left(\dfrac{a_{ux} \pi x}{L}\right)+2 a_{vy} v_y \cos\left(\dfrac{a_{vy} \pi y}{L}\right)\right] .
\end{split}
\end{equation}
where  $\U_\text{trans}$ and $\V_\text{trans}$ are defined in  Equation (\ref{aux_var_transient}).

\section{Steady Inviscid Burgers Equation}

The 2D steady inviscid Burgers  equations are:
\begin{equation}
 \label{eq:burgers2d_swv}
\begin{split}
&\diff{ u^2 }{x}+\diff{uv}{y}=0,\\
& \diff{ u v}{x} + \diff{  v^2 }{y}=0.
\end{split}
\end{equation}

%
Analogously to the steady viscous case, source terms $Q_u$ and $Q_v$ for the steady inviscid Burgers equations are  obtained by using the time-independent  manufactured solution (\ref{eq:manufactured03}):
\begin{equation}
\begin{split}\label{sourceQu_t}
Q_u = &-\dfrac{a_{uy} \pi u_y \V_\text{steady} }{L}\sin\left(\dfrac{a_{uy} \pi y}{L}\right)+\\
 &+\dfrac{\pi \U_\text{steady}}{L}\left[2 a_{ux} u_x \cos\left(\dfrac{a_{ux} \pi x}{L}\right)+a_{vy} v_y \cos\left(\dfrac{a_{vy} \pi y}{L}\right)\right].
\end{split}
 \end{equation}
and
\begin{equation}
\begin{split}
Q_v =  &-\dfrac{a_{vx} \pi v_x \U_\text{steady} }{L}\sin\left(\dfrac{a_{vx} \pi x}{L}\right)+\\
 &+\dfrac{\pi \V_\text{steady}}{L}\left[a_{ux} u_x \cos\left(\dfrac{a_{ux} \pi x}{L}\right)+2 a_{vy} v_y \cos\left(\dfrac{a_{vy} \pi y}{L}\right)\right] .
\end{split}
\end{equation}
where  $\U_\text{steady}$ and $\V_\text{steady}$ are defined in  Equation (\ref{aux_var_steady}).






\section{Comments}

Source terms $Q_u$ and $Q_v$ for the distinct regimen of Burgers equations have been generated by replacing the either the transient or the steady version of the analytical  manufactured solutions into the corresponding set of Equations (\ref{eq:burgers2d_tv}), (\ref{eq:burgers2d_sv}), (\ref{eq:burgers2d_twv}) and/or (\ref{eq:burgers2d_swv}), followed by the usage of Maple commands for collecting, sorting and factorizing the terms. Files containing $C$ codes for the source terms have also been generated:  \texttt{Burgers\_2d\_u\_code\_all\_cases.C} and \texttt{Burgers\_2d\_v\_code\_all\_cases.C,}

Additionally to verifying code capability of solving the governing equations accurately in the interior of the domain of interest, one may also verify the software's capability of correctly imposing boundary conditions. Therefore, the gradients of the  analytical solutions (\ref{eq:manufactured01}) have been calculated:
\begin{equation*}
\nabla u = \left[ \begin{array}{c}
  \dfrac{  a_{ux}  \pi u_x}{L} \cos\left( \dfrac{ a_{ux}  \pi  x}{L}\right)\vspace{5pt}\\
 -   \dfrac{  a_{uy}  \pi u_y}{L} \sin\left( \dfrac{ a_{uy}  \pi  y}{L}\right)
\end{array} \right],
\quad\mbox{and}\quad
\nabla  v= \left[ \begin{array}{c}
-  \dfrac{  a_{vx}  \pi v_x}{L}  \sin\left( \dfrac{ a_{vx}  \pi  x}{L}\right)\vspace{5pt}\\
    \dfrac{  a_{vy}  \pi v_y}{L} \cos\left( \dfrac{ a_{vy}  \pi  y}{L}\right)
\end{array} \right],
\end{equation*}
translated into  $C$ code  and stored in the file \texttt{Burgers\_2d\_manuf\_solutions\_grad\_code.C}.


An example of the automatically generated C file from the source term for velocity $u$ for the transient viscous flow (Equation (\ref{sourceQu_complete})) is:
\begin{small}
\begin{verbatim}
double SourceQ_u_transient_viscous (double x, double y, double t, double nu)
{
  double Qu_tv;
  double U;
  double V;

  U = u_0 + u_x * sin(a_ux * PI * x / L) + u_y * cos(a_uy * PI * y / L) + u_t * cos(a_ut * PI * t / L);
  V = v_0 + v_x * cos(a_vx * PI * x / L) + v_y * sin(a_vy * PI * y / L) + v_t * sin(a_vt * PI * t / L);

  Qu_tv = -a_ut * PI * u_t * sin(a_ut * PI * t / L) / L
    - a_uy * PI * u_y * V * sin(a_uy * PI * y / L) / L
    + (0.2e1 * a_ux * u_x * cos(a_ux * PI * x / L) + a_vy * v_y * cos(a_vy * PI * y / L)) * PI * U / L
    + a_ux * a_ux * PI * PI * u_x * nu * sin(a_ux * PI * x / L) * pow(L, -0.2e1)
    + a_uy * a_uy * PI * PI * u_y * nu * cos(a_uy * PI * y / L) * pow(L, -0.2e1);
  return(Qu_tv);
}
\end{verbatim}
\end{small}


%---------------------------------------------------------------------------------------------------------
\bibliographystyle{chicago} 
\bibliography{/home/kemelli/MMS_maple_workplace/heat_equation/MMS_bib}

\end{document}


\begin{equation}
\nabla \rho = \left[ \begin{array}{c}
 \dfrac{  a_{\rho x}  \pi rho_x }{L} \cos\left( \dfrac{ a_{\rho x}  \pi  x }{L}\right) \vspace{5pt}\\
 -\dfrac{  a_{\rho y}  \pi rho_y }{L} \sin\left( \dfrac{ a_{\rho y}  \pi  y }{L}\right)
\end{array} \right]\qquad
\end{equation}


\begin{equation}
\nabla p = \left[ \begin{array}{c}
- \dfrac{  a_{px}  \pi p_x }{L} \sin\left( \dfrac{ a_{px}  \pi  x }{L}\right) \vspace{5pt}\\
  \dfrac{  a_{py}  \pi p_y }{L} \cos\left( \dfrac{ a_{py}  \pi  y }{L}\right)
\end{array} \right]
\end{equation}


\begin{equation}
\nabla u = \left[ \begin{array}{c}
\dfrac{  a_{ux}  \pi u_x}{L} \cos\left( \dfrac{ a_{ux}  \pi  x }{L}\right)\vspace{5pt} \\
-  \dfrac{  a_{uy}  \pi u_y }{L} \sin\left( \dfrac{ a_{uy}  \pi  y }{L}\right)
\end{array} \right]
\end{equation}


\begin{equation}
\nabla v = \left[ \begin{array}{c}
-  \dfrac{  a_{vx}  \pi v_x }{L} \sin\left( \dfrac{ a_{vx}  \pi  x }{L}\right)\vspace{5pt} \\
 \dfrac{  a_{vy}  \pi  v_y }{L} \cos\left( \dfrac{ a_{vy}  \pi  y }{L}\right)
\end{array} \right]
\end{equation}
