\documentclass[10pt]{article}
\usepackage[utf8x]{inputenc}
\usepackage{amsmath}
\usepackage{geometry}
\geometry{ top=2.5cm, bottom=2cm, left=2cm, right=2cm}
\usepackage[authoryear]{natbib}
\usepackage{pdflscape}
%\geometry{papersize={216mm,330mm}, top=3cm, bottom=2.5cm, left=4cm,  right=2cm}

\newcommand{\D}{\partial}
\newcommand{\Diff}[2] {\dfrac{\partial( #1)}{\partial #2}}

%opening
\title{Manufactured Solution for 1D Euler equation using Maple\footnote{Work based on \citet*{Roy2002}.}}
\author{Kemelli C. Estacio-Hiroms}

\begin{document}

\maketitle

\begin{abstract}
The Method of Manufactured Solutions is a valuable approach for code verification, providing means to verify how accurately the numerical method solves the equations of interest. The method generates a related set of governing equations by adding a source term to the RHS of the original set of equations, making use of analytical solutions chosen a priori. 
This document presents the source terms generated by the application of the Method of Manufactured Solutions (MMS) on the the 1D Euler equations using the analytical manufactured solutions for $\rho,\, u$ and $p$ presented by \citet{Roy2002}.
\end{abstract}





\section{1D Euler Equations}
The 1D Euler equations in conservation form are:
\begin{equation}
 \label{eq:euler2d_01}
\Diff{\rho}{t} + \Diff{\rho u}{x} = 0,
\end{equation}


\begin{equation}
 \label{eq:euler2d_02}
\Diff{\rho u}{t} + \Diff{\rho u^2 + p}{x} = 0,
\end{equation}




\begin{equation}
 \label{eq:euler2d_04}
\Diff{\rho e_t}{t} + \Diff{\rho ue_t +pu}{x} = 0,
\end{equation}
%
where the Equation (\ref{eq:euler2d_01}) is the unsteady term (mass conservation), Equation (\ref{eq:euler2d_02}) represents the nonlinear convection term in the $x$  direction (momentum), and Equation (\ref{eq:euler2d_04}) is the energy. For a calorically perfect gas, the Euler equations are closed with two auxiliary relations for energy:
\begin{equation}
 \label{eq:euler2d_05}
e=\dfrac{1}{\gamma -1}RT ,
\end{equation}
%
\begin{equation}
 \label{eq:euler2d_06}
e_t= e+\dfrac{u^2}{2},
\end{equation}
and with the ideal gas equation of state:
\begin{equation}
 \label{eq:euler2d_07}
p=\rho RT.
\end{equation}

\section{Manufactured Solution}

\citet{Roy2002} propose the general form of the primitive solution variables to be  a function of sines and cosines, which to the one-dimensional case is reduced to:
\begin{equation}
 \label{eq:manufactured01}
  \phi (x) = \phi_0+ \phi_x f_s\left(\frac{a_{\phi x} \pi x}{L}\right),
\end{equation}
where $\phi=\rho, \,u$ or $p$, and $f_s(\cdot)$ functions denote either sine or cosine function. Note that in this case, $\phi_x$ is a constant  and the subscript do not denote differentiation.

Therefore, the manufactured analytical solution for for each one of the variables in Euler equations are:
\begin{equation}
\begin{split}
\label{eq:manufactured02}
\rho\left(x,y\right) &=  \rho_{0}+ \rho_{x} \sin\left(\frac{a_{ \rho x} \pi x}{L}\right),\\
u\left(x,y\right) &= u_{0}+u_{x} \sin\left(\frac{a_{u x} \pi x}{L}\right),\\
p\left(x,y\right) &= p_{0}+p_{x} \cos\left(\frac{a_{p x} \pi x}{L}\right).\\
\end{split}
\end{equation}


The MMS applied to Euler equations consists in modifying Equations~(\ref{eq:euler2d_01}) -- (\ref{eq:euler2d_04}) by adding a source term to the right-hand side of each equation:
\begin{equation}
 \label{eq:euler_mod}
\begin{split}
\Diff{\rho}{t} + \Diff{\rho u}{x} &= Q_\rho,\\
\Diff{\rho u}{t} + \Diff{\rho u^2 + p}{x} &= Q_u,\\
\Diff{\rho e_t}{t} + \Diff{\rho ue_t +pu}{x} &= Q_e,
\end{split}
\end{equation}
%
so the modified set of equations (\ref{eq:euler_mod}) conveniently has the analytical solution given in Equation (\ref{eq:manufactured02}). This is achieved by simply applying Equations~(\ref{eq:euler2d_01}) -- (\ref{eq:euler2d_04}) as operators on Equation (\ref{eq:manufactured01}).
%
Terms $Q_\rho$, $Q_u$,   and $Q_{e}$ are obtained by symbolic manipulations of equations above using Maple and are presented in the following sections.



\section{Euler mass conservation equation}

The mass conservation equation written as an operator is:
\begin{equation}
 \label{eq:euler2d_11}
L= \Diff{\rho}{t} + \Diff{\rho u}{x}.
\end{equation}

Analytically differentiating Equation (\ref{eq:manufactured02}) for $\rho$ and $u$  using operator $L$ defined above gives  the source term $Q_{\rho}$:
\begin{equation}
\begin{split}
Q_\rho &=\dfrac{a_{\rho x} \pi \rho_x }{L} \cos\left( \dfrac{a_{\rho x} \pi x}{L} \right) \left[ u_{0}+u_{x} \sin\left(\frac{a_{u  x} \pi x}{L}\right) \right] +\\
&+\dfrac{a_{ux} \pi u_x }{L} \cos\left( \dfrac{a_{ux} \pi x}{L} \right) \left[  \rho_{0}+ \rho_{x} \sin\left(\frac{a_{ \rho  x} \pi x}{L}\right) \right] 
.
\end{split}
\end{equation}


\section{Euler momentum equation}

For the generation of the analytical source term $Q_u$ for the $x$ momentum equation, Equation  (\ref{eq:euler2d_02}) is written as an  operator $L$:
\begin{equation}
 \label{eq:euler2d_12}
L=\Diff{\rho u}{t} + \Diff{\rho u^2 + p}{x},
\end{equation}
which, when operated in Equation (\ref{eq:manufactured02}), provides source term $Q_{u}$:

\begin{equation}
\begin{split}
Q_u &=  
\dfrac{a_{\rho x} \pi \rho_x }{L}\cos\left( \dfrac{a_{\rho x} \pi x}{L} \right) \left[ u_{0}+u_{x} \sin\left(\frac{a_{u  x} \pi x}{L}\right) \right]^2  +\\
&-\dfrac{a_{px} \pi p_x }{L} \sin\left( \dfrac{a_{px} \pi x}{L} \right)  +\\
&+\dfrac{2a_{ux} \pi u_x }{L}\cos\left( \dfrac{a_{ux} \pi x}{L} \right) \left[  \rho_{0}+ \rho_{x} \sin\left(\frac{a_{ \rho  x} \pi x}{L}\right) \right] \left[ u_{0}+u_{x} \sin\left(\frac{a_{u  x} \pi x}{L}\right) \right].
\end{split}
 \end{equation}



\section{Euler energy equation}


The last component of Euler equations is written as an operator:
\begin{equation}
 \label{eq:euler2d_14}
L=\Diff{\rho e_t}{t} + \Diff{\rho ue_t +pu}{x} .
\end{equation}


Source term $Q_e$ is obtained by operating $L$ on Equation  (\ref{eq:manufactured02}) together with the use of the  auxiliary relations~(\ref{eq:euler2d_05})--(\ref{eq:euler2d_07}) for energy :
  \begin{equation}\label{eq:source_e}
 \begin{split}
\displaystyle
Q_e &= \dfrac{a_{\rho x} \pi \rho_x }{2L}  \cos\left( \dfrac{a_{\rho x} \pi x}{L} \right) \left[ u_{0}+u_{x} \sin\left(\frac{a_{u  x} \pi x}{L}\right) \right]^3 +\\
&-\dfrac{a_{px} \pi p_x }{L}\dfrac{\gamma}{\gamma-1}\sin\left( \dfrac{a_{px} \pi x}{L} \right) \left[ u_{0}+u_{x} \sin\left(\frac{a_{u  x} \pi x}{L}\right) \right] +\\
&+\dfrac{a_{ux} \pi u_x }{L}\dfrac{\gamma}{\gamma-1}\cos\left( \dfrac{a_{ux} \pi x}{L} \right) \left[ p_{0}+p_{x} \cos\left(\frac{a_{p  x} \pi x}{L}\right)\right] +\\ 
&+ \dfrac{3a_{ux} \pi u_x }{2L}  \cos\left( \dfrac{a_{ux} \pi x}{L} \right)\left[  \rho_{0}+ \rho_{x} \sin\left(\frac{a_{ \rho  x} \pi x}{L}\right) \right] \left[ u_{0}+u_{x} \sin\left(\frac{a_{u  x} \pi x}{L}\right) \right]^2.
 \end{split}
 \end{equation}

 

\section{Comments}

Source terms $Q_{\rho}$, $Q_u$ and $Q_e$ have been generated automatically by replacing the analytical expressions (\ref{eq:manufactured02}) into  respective operators  (\ref{eq:euler2d_11}), (\ref{eq:euler2d_12}) and (\ref{eq:euler2d_14}), followed by the usage of Maple commands for collecting, sorting and factorizing the terms. 
 

One file containing  $C$ codes for the source terms has also been generated:  \texttt{Euler\_1d\_e\_codes.C}. As an example, the automatically generated C file from the source term for the total energy is:
\begin{verbatim}
double SourceQ_e (double x, double y, double u_0, double u_x, double rho_0, double rho_x,
                  double p_0, double p_x, double a_px, double a_rhox, double a_ux,
                  double Gamma, double mu, double L)
{
  double Q_e;
  Q_e = cos(a_rhox * PI * x / L) * rho_x * pow(u_0 + u_x * sin(a_ux * PI * x / L), 0.3e1) *
        a_rhox * PI / L / 0.2e1 + cos(a_ux * PI * x / L) * (p_0 + p_x * cos(a_px * PI * x / L)) *
        a_ux * PI * u_x * Gamma / L / (Gamma - 0.1e1) - Gamma * p_x * sin(a_px * PI * x / L) *
        (u_0 + u_x * sin(a_ux * PI * x / L)) * a_px * PI / L / (Gamma - 0.1e1) + 0.3e1 / 0.2e1 *
        cos(a_ux * PI * x / L) * (rho_0 + rho_x * sin(a_rhox * PI * x / L)) *
        pow(u_0 + u_x * sin(a_ux * PI * x / L), 0.2e1) * a_ux * PI * u_x / L;
  return(Q_e);
}
\end{verbatim}

Finally the gradients of the analytical solutions have also been computed and their respective C codes are presented in \texttt{Euler\_manuf\_solutions\_grad\_and\_code\_1d.C}. Therefore, the gradients of the analytical solution~(\ref{eq:manufactured02}):
\begin{equation}
\nabla \rho =  \dfrac{a_{\rho x} \pi \rho_x}{L}  \cos\left( \dfrac{a_{\rho x} \pi x}{L} \right),
\qquad
\nabla p = -\dfrac{a_{px} \pi p_x }{L} \sin\left( \dfrac{a_{px} \pi x}{L} \right) ,
\qquad
\nabla u =  \dfrac{a_{ux} \pi u_x }{L}  \cos\left( \dfrac{a_{ux} \pi x}{L} \right),
\end{equation}
are written in C language as:

\begin{verbatim}
grad_rho_an[0] = rho_x * cos(a_rhox * pi * x / L) * a_rhox * pi / L;
grad_rho_an[1] = 0;
grad_rho_an[2] = 0;
grad_p_an[0] = -p_x * sin(a_px * pi * x / L) * a_px * pi / L;
grad_p_an[1] = 0;
grad_p_an[2] = 0;
grad_u_an[0] = u_x * cos(a_ux * pi * x / L) * a_ux * pi / L;
grad_u_an[1] = 0;
grad_u_an[2] = 0;
\end{verbatim}

%---------------------------------------------------------------------------------------------------------
\bibliographystyle{chicago} 
\bibliography{../../MMS_bib}

\end{document}


\begin{equation}
\nabla \rho = \left[ \begin{array}{c}
 \dfrac{  a_{\rho x}  \pi rho_x }{L} \cos\left( \dfrac{ a_{\rho x}  \pi  x }{L}\right) \vspace{5pt}\\
 -\dfrac{  a_{\rho y}  \pi rho_y }{L} \sin\left( \dfrac{ a_{\rho y}  \pi  y }{L}\right)
\end{array} \right]\qquad
\end{equation}


\begin{equation}
\nabla p = \left[ \begin{array}{c}
- \dfrac{  a_{px}  \pi p_x }{L} \sin\left( \dfrac{ a_{px}  \pi  x }{L}\right) \vspace{5pt}\\
  \dfrac{  a_{py}  \pi p_y }{L} \cos\left( \dfrac{ a_{py}  \pi  y }{L}\right)
\end{array} \right]
\end{equation}


\begin{equation}
\nabla u = \left[ \begin{array}{c}
\dfrac{  a_{ux}  \pi u_x}{L} \cos\left( \dfrac{ a_{ux}  \pi  x }{L}\right)\vspace{5pt} \\
-  \dfrac{  a_{uy}  \pi u_y }{L} \sin\left( \dfrac{ a_{uy}  \pi  y }{L}\right)
\end{array} \right]
\end{equation}


\begin{equation}
\nabla v = \left[ \begin{array}{c}
-  \dfrac{  a_{vx}  \pi v_x }{L} \sin\left( \dfrac{ a_{vx}  \pi  x }{L}\right)\vspace{5pt} \\
 \dfrac{  a_{vy}  \pi  v_y }{L} \cos\left( \dfrac{ a_{vy}  \pi  y }{L}\right)
\end{array} \right]
\end{equation}
