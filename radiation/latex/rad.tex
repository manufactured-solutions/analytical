\documentclass{article}
\title{\bf{Radiation Verification Document}}
\author{Nicholas Malaya, Marco Panesi \\ Institute for Computational Engineering and Sciences \\ University of Texas at Austin} \date{}

\begin{document}
\maketitle

In order to verify the integration in the spectral radiation code, we will develop an analytical solution for 
the summation of the components of several gaussian distributions, over the interval $[0,\infty]$. 
\newline
\newline
In other words, we seek to calculate:
\begin{equation}
  \sum_i^n \int^\infty_0 g_i(x) dx
\end{equation}
\newline
\newline
$g_i(x)$ is a set of $i$ gaussian distributions, each with distinct mean ($\mu_i$), standard deviation ($\sigma_i$), and amplitude ($A_i$):
\begin{equation}
  g_i(x) = \frac{A_i}{\sqrt{(2 \pi)}}e^{\frac{-(x_i-\mu_i)^2}{2 \sigma_i^2}}
\end{equation}
\newline
\newline
Note that,
\begin{equation}
  \int^\infty_0 g(x) dx = \int^\infty_\infty g(x) dx  -  \int^0_{-\infty} g(x)dx
\end{equation}
In addition we can easily calculate the cumulative distribution function for a gaussian:
\begin{equation}
  \int^x_{-\infty} g(x) dx = A_i * \frac12 [1+\textrm{erf}(\frac{x_i-\mu_i}{\sigma_i})] = A_i * \Phi(\frac{x_i-\mu_i}{\sigma_i})
\end{equation}
Thus, 
\begin{equation}
  \sum_i^n \int^\infty_0 g_i(x) dx = \sum_i^n ( \int^\infty_{-\infty} g_i(x) dx  -  \int^0_{-\infty} g_i(x)dx)
\end{equation}
and,
\begin{equation}
  \sum_i^n \int^\infty_0 g_i(x) dx = \sum_i^n ( A_i  - A_i * \frac12 [1+\textrm{erf}(-\frac{\mu_i}{\sigma_i})])
\end{equation}
We now arrive at our solution:
\begin{equation}
  \sum_i^n \int^\infty_0 g_i(x) dx = \sum_i^n A_i ( 1  -  \Phi(-\frac{\mu_i}{\sigma_i}))
\end{equation}
\newline
\newline
Additionally, if a user wanted to set the bounds from $[0,\infty]$ to $[0,x]$:
\begin{equation}
  \sum_i^n \int^x_0 g_i(x) dx = \sum_i^n A_i ( \Phi(\frac{x_i-\mu_i}{\sigma_i}) - \Phi(-\frac{\mu_i}{\sigma_i}))
\end{equation}


\end{document}
